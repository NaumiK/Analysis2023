\documentclass[a4paper,12pt]{article}


\usepackage{cmap} % Поиск в PDF
\usepackage[utf8]{inputenc} % Кодировка исходников
\usepackage[english, russian]{babel} % Кодировка 

\usepackage{amsmath, amssymb, amsfonts, amsthm, mathtools} % AMS packages
\usepackage{icomma}

\newcommand{\Name}{Математический анализ}
\newcommand{\Lecturer}{Заботин Владислав Иванович}
\newcommand{\Date}{2023}


\theoremstyle{plain}
\newtheorem{theorem}[equation]{Теорема}
\newtheorem{lemma}[equation]{Лемма}
\newtheorem*{statement}{Утверждение}
\newtheorem*{problem}{Задача}
\newtheorem*{example}{Пример}
\theoremstyle{definition}
\newtheorem*{definition}{Определение}
\newtheorem*{note}{Замечание}
\newtheorem*{dsg}{\underline{Обозначение}}
\newtheorem*{prp}{\underline{Свойства}}

\newcommand{\R}{\mathbb{R}}
\newcommand{\N}{\mathbb{N}}
\newcommand{\Cx}{\mathbb{C}}
\newcommand{\Q}{\mathbb{Q}}
\newcommand{\V}{\mathbb{V}}
\newcommand{\K}{\mathbb{K}}
\newcommand{\lsq}{\leqslant}
\newcommand{\gsq}{\geqslant}
\newcommand{\eps}{\varepsilon}
\newcommand{\eqt}[1]{\stackrel{\mathrm{#1}}{=}}
\newcommand{\mbb}[1]{\mathbb{#1}}
\newcommand{\mum}[3]{\sum \limits_{#1}^{#2} {#3}} % My sUM
\newcommand{\mim}[2]{\lim \limits_{#1} #2} % My lIM
\newcommand{\bij}{\text{BIJ}}
\newcommand{\abs}[1]{\left\lvert #1 \right\rvert}
\newcommand{\fsb}[1]{\left\{ #1 \right\}} % figure sizeable brackets 
\newcommand{\seg}[1]{\left[ #1 \right]}
\newcommand{\ivl}[1]{\left( #1 \right)}
\newcommand{\isconst}{=\text{const}}
\newcommand{\const}{\text{const}}
\newcommand{\dlta}{\text{d}}
\newcommand{\floor}[1]{\left\lfloor #1 \right\rfloor}
\newcommand{\ceil}[1]{\left\lceil #1 \right\rceil}
\newcommand{\none}{\varnothing}
\newcommand{\ol}[1]{\overline{#1}}
\newcommand{\tif}{\text{если }} % Text IF

\DeclareMathOperator{\sgn}{sgn}
\DeclareMathOperator{\ke}{Ker}
\DeclareMathOperator{\diag}{diag}
\DeclareMathOperator{\im}{Im}
\DeclareMathOperator{\re}{Re}

\usepackage{listings}
\usepackage{color}

\definecolor{dkgreen}{rgb}{0,0.6,0}
\definecolor{gray}{rgb}{0.5,0.5,0.5}
\definecolor{mauve}{rgb}{0.58,0,0.82}

\lstloadlanguages{[ISO]C++, Python}
\lstset{ 
    language=Python,
    extendedchars=\true,
    escapechar=|,
    frame=tb,
    commentstyle=\itshape,
    stringstyle=\bfseries,
    numbers=left,
    keywordstyle=\color{blue},
    commentstyle=\color{dkgreen},
    stringstyle=\color{mauve},
    tabsize=4
    }
    
\usepackage{geometry} % Меняем поля страницы
\geometry{left=2cm}% левое поле
\geometry{right=2cm}% правое поле
\geometry{top=2.5cm}% верхнее поле
\geometry{bottom=3cm}% нижнее поле

\usepackage[usenames,dvipsnames,svgnames,table,rgb]{xcolor}
\usepackage{graphicx}
\usepackage{tikz}
\usepackage{pgfplots}
\usetikzlibrary{calc}
% \usepackage{pstricks}
\usepackage{transparent}
% \usepackage{auto-pst-pdf}
\graphicspath{ {./images/} }

%%% Работа с таблицами
\usepackage{array,tabularx,tabulary,booktabs} % Дополнительная работа с таблицами
\usepackage{longtable}                        % Длинные таблицы
\usepackage{multirow}                         % Слияние строк в таблице


\usepackage{hyperref}
\hypersetup{
	unicode=true,            % русские буквы в раздела PDF
	colorlinks=true,       	 % Цветные ссылки вместо ссылок в рамках
	linkcolor=black!15!blue, % Внутренние ссылки
	citecolor=green,         % Ссылки на библиографию
	filecolor=magenta,       % Ссылки на файлы
	urlcolor=NavyBlue,       % Ссылки на URL
}

% Изменить колонтинулы!!!
\usepackage{titleps}
\newpagestyle{main}{
	\setheadrule{0.4pt}
	\sethead{\Name}{}{\Lecturer}
	\setfootrule{0.4pt}                       
	\setfoot{КАИ ИКТЗИ ПМИ, \Date}{}{\thepage} 
}
\pagestyle{main}  

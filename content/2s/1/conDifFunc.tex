\subsection{Непрерывно-дифференцируемая функция}
\begin{definition} 
Функция, в окрестности точки \( x_0 \) которой определена производная \( n \)-го порядка, непрерывная в этой точке --- \( n \) раз непрерывно-дифференцируемая функция в точке \( x_0 \).
\end{definition}
Не каждая функция является непрервыно-дифференцируемой на всей области определения.
\begin{example} Функция, непрерывно-дифференцируемая в точке \( 0 \): 
\[ 
    f(x) = \left\{ \begin{array}{ll}
        x^2, & x \in \Q; \\
        0,   & x \in \R \setminus \Q.
    \end{array} \right.
\]
\end{example}

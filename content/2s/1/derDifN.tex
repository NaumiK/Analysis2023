\subsection{Производные и дифференциалы высших порядков}
\begin{definition}
    Пусть функция \( f \) дифференцируема на \( \ivl{a, b} \ni x_0 \), тогда определена функция \( \varphi\colon \ivl{a, b} \to \R \colon x \mapsto f'(x) \). Если \( \varphi \) дифференцируема в точке \( x_0 \), то \( \varphi'(x_0) \) - производная 2-го порядка.
\end{definition}
\begin{dsg} 
    \[ f''(x_0) := \varphi'(x_0) =: \frac{\dlta^2f(x_0)}{\dlta x^2}\]
\end{dsg}
\begin{example}
    \[ f(x)  = \sin(x) (x \in \R) \]
    \[ f'(x) = \cos(x) (x \in \R) \]
    \[ f''(x)=-\sin(x) (x \in \R) \]
\end{example}
\begin{definition} 
    Выражение вида \( \dlta^2f(x_0) := f''(x_0)(\dlta x)^2 = f''(x_0)\dlta x^2 \) -- дифференциал 2-го порядка функции \( f \) в точке \( x_0 \). Аналогично можно ввести дифференциал \( n \)-го порядка: 
    \[ \dlta^nf(x_0) = f^{(n)}(x_0)(\dlta x)^n, \]
    где \( f^{(n)}(x_0) \) -- производная \( n \)-го порядка.
\end{definition}
\begin{example} 
    Общая функция производной \( \sin \):
    \[ f(x) = \sin(x) \]
    \[ f'(x) = \cos(x) = \sin\ivl{x + \frac{\pi}{2}} \]
    \[ f''(x) = \ivl{\sin\ivl{x + \frac{\pi}{2}}}' = \cos\ivl{x + \frac{\pi}{2}}\ivl{x+\frac{\pi}{2}}' = \cos\ivl{x+\frac{\pi}{2}} = -\sin(x) \]
    \[ f'''(x) = \sin{\ivl{x + 3\frac{\pi}{2}}} \]
    \[ f^{(n)}(x)=\sin{\ivl{x+n\frac{\pi}{2}}} \]
    Где \( f^{(n)}(x) \) -- производная \( n \)-го порядка.
\end{example}
\begin{problem} 
    Найти производную \( n \)-го порядка для функции \( f(x) = \ln(x) \), где \( x > 0 \).
\end{problem}
Сперва рассмотрим \( f', f'', f''' \):
\[ f'(x) = \frac{1}{x} = x^{-1} \]
\[ f''(x) = \ivl{x^{-1}}' = (-1)x^{-1-1} = -x^{-2} \]
\[ f'''(x) = \ivl{-x^{-2}}' = 2x^{-3} = \frac{2}{x^3} \]
Итак, мы здесь видим закономерность: находя производную вновь и вновь, мы <<спускаем>> предыдущий, \( (n-1) \)-й коэффициент, степень при x увеличивается, всё выражение меняет знак. Так, мы получаем выражение:
\begin{equation} \label{ExLnDer}
    f^{(n)}(x) = \left\{ \begin{array}{ll}
        (-1)^{n-1}\frac{(n-1)!}{x^n}, & n > 0;\\
        \ln(x), &  n = 0.
    \end{array} \right.
\end{equation}
\begin{problem} 
    Проверить правдивость выражения \ref{ExLnDer} для \( f(x) = \ln(x), x > 0 \) 
\end{problem}
\begin{proof}
    Проверим по индукции: 
    Пусть \( E = \fsb{n \in \N \mid P \footref{ExLnDer}(n) } \)
    \begin{enumerate}
        \item \( E \subset \N \);
        \item \( f'(x) = \frac{1}{x} = (-1)^0 \frac{0!}{x^1}  \Longrightarrow 1 \in E \);
        \item Пусть \( n \in E \Longrightarrow f^{(n)}(x) = (-1)^n \frac{(n-1)!}{x^n} \);
        \item Докажем, что \( (n + 1) \in E \): 
        \begin{multline*} 
            f^{(n+1)}(x) = \ivl{f^{(n)}}' = \ivl{(-1)^{n-1} \frac{(n-1)!}{x^n}}' = (-1)^{n-1}(n-1)!\ivl{x^{-n}}' = \\ 
            = (-1)^{n-1}(n-1)!(-n)\ivl{x^{-n-1}} = (-1)^{n}n!\ivl{x^{-n-1}} = (-1)^n \frac{n!}{x^{n+1}} = f^{(n+1)}(x) 
        \end{multline*}
        \item Итак, \( n \in E \Longrightarrow (n+1) \in E \);
        \item Итак: \( \seg{E \subset \N \land 1 \in E \land n \in E \Longrightarrow (n+1) \in E} \Longrightarrow E = \N \)
    \end{enumerate}
    Итак, для любого натурального числа выполняется \ref{ExLnDer}, \( f(x) = \ln(x) = f^{(0)}(x) \), поэтому выражение \ref{ExLnDer} является правдой.
\end{proof}
\begin{note}
    Для нахождения производной \( n \)-го порядка в точке \( x_0 \) мы должны иметь производную \( (n-1) \)-го порядка, определённую в окрестности этой точки \( U(x_0) \)
\end{note}

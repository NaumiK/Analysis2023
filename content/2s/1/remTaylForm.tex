\subsection{Остаточный член формулы Тейлора}
\begin{theorem} 
    Об остатке формулы Тейлора \( R_n \) в форме Пеано. \\
    Пусть функция \( f \) \( n \) раз дифференцируема в точке \( x_0 \), тогда: \( R_n(x) = o((x-x_0)^n) \).
\end{theorem}
\begin{proof} 
    Функция \( f \) \( n \) раз дифференцируема, так что мы можем рассмотреть следующее отношение:
    \begin{multline} \label{RnToxxn}
        \frac{R_n(x)}{(x-x_0)^n} = \frac{f(x)-T_n(x)}{(x-x_0)^n} = \frac{f(x) - f(x_0) - f'(x_0)(x-x_0) - \dots - \frac{f^{(n)}(x_0)}{n!}}{(x-x_0)^n} = \\
        = \ivl{\frac{0}{0}, x \to x_0} \sim \frac{f'(x) - f'(x_0) - 2f''(x_0)(x-x_0)-\frac{3}{2}f'''(x_0)(x-x_0)^2 -\dots}{n(x-x_0)^{n-1}} = \\
        = \ivl{\frac{0}{0}, x \to x_0} \sim \dots \text{n-1 раз} \sim \frac{f^{(n-1)}(x) - f^{(n-1)}(x_0) - f^{(n)}((x-x_0))}{n!(x-x_0)}
    \end{multline}
    Всё это время мы применяли 2-е правило Лопиталя \( (n - 1) \) раз, но тут столкнулись с трудностью: у нас нет достаточных условий для применения 2-го правила вновь: мы имеем \( f^{(n)} \) лишь в точке \( x_0 \), не в окрестности \( U(x_0) \) - следовательно нельзя взять \( \mim{x \to x_0}{\frac{f'(x)}{g'(x)}} \). \\
    Но при том мы можем воспользоваться 1-м правилом Лопиталя и тогда:
    \[ (\refeq{RnToxxn}) \longrightarrow \frac{f^{(n)}(x_0) - f^{(n)}(x_0)}{n!} = 0 \]
\end{proof}